\chapter*{Abstract}

Disk phenomena appear in many astrophysical systems and interact with
almost any kind of objects in the Universe.
Understanding the dynamics of disks is one of the most important areas
of study in modern astrophysics.
Although theoretical study of accretion disks have been carried out
for half a century, we have not yet achieved a self-consistent
understanding.
The major difficulties of accretion theory comes from the
non-linearity of magnetohydrodynamics as well as the large degree of
freedom in turbulent flows, where analytical studies bring limited
progress.
This dissertation is about developing new numerical algorithms based
on analytical understanding in order to describe the dynamics of
accretion disks and to study astrophysical turbulence.
I will present a new pseudo-spectral algorithm that we developed to
study turbulence as well as the stability and general properties of
accretion disks.
I will show the strengths of this new algorithm in comparison to other
previously developed ones.
I will also present the results from applications of these algorithms
to several problems involving accretion disks.
